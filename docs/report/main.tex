\documentclass[DIV=calc, paper=a4, fontsize=11pt, twocolumn]{scrartcl}

\usepackage{lipsum} % Used for inserting dummy 'Lorem ipsum' text into the template
\usepackage[english]{babel} % English language/hyphenation
\usepackage[protrusion=true,expansion=true]{microtype} % Better typography
\usepackage{amsmath,amsfonts,amsthm} % Math packages
\usepackage[svgnames]{xcolor} % Enabling colors by their 'svgnames'
\usepackage[hang, small,labelfont=bf,up,textfont=it,up]{caption} % Custom captions under/above floats in tables or figures
\usepackage{booktabs} % Horizontal rules in tables
\usepackage{fix-cm}	 % Custom font sizes - used for the initial letter in the document

\usepackage{sectsty} % Enables custom section titles
\allsectionsfont{\usefont{OT1}{phv}{b}{n}} % Change the font of all section commands

\usepackage{fancyhdr} % Needed to define custom headers/footers
\pagestyle{fancy} % Enables the custom headers/footers
\usepackage{lastpage} % Used to determine the number of pages in the document (for "Page X of Total")
\usepackage[colorlinks=false]{hyperref}
\usepackage{paralist}

% Headers - all currently empty
\lhead{}
\chead{}
\rhead{}

% Footers
\lfoot{}
\cfoot{}
\rfoot{\footnotesize Page \thepage\ of \pageref{LastPage}} % "Page 1 of 2"

\renewcommand{\headrulewidth}{0.0pt} % No header rule
\renewcommand{\footrulewidth}{0.4pt} % Thin footer rule

\usepackage{lettrine} % Package to accentuate the first letter of the text
\newcommand{\initial}[1]{ % Defines the command and style for the first letter
\lettrine[lines=3,lhang=0.3,nindent=0em]{
\color{DarkGoldenrod}
{\textsf{#1}}}{}}

%----------------------------------------------------------------------------------------
%	TITLE SECTION
%----------------------------------------------------------------------------------------

\usepackage{titling} % Allows custom title configuration

\newcommand{\HorRule}{\color{DarkGoldenrod} \rule{\linewidth}{1pt}} % Defines the gold horizontal rule around the title

\begin{document}

\pretitle{\vspace{-50pt} \begin{flushleft} \HorRule \fontsize{40}{40} \usefont{OT1}{phv}{b}{n} \color{DarkRed} \selectfont} % Horizontal rule before the title

\title{{\large A GSoC 2013 Final Report}\\Enhancing Giri: \\Dynamic Slicing in LLVM}

\posttitle{\par\end{flushleft}\vskip 0.5em} % Whitespace under the title

\preauthor{\begin{flushleft}\large \lineskip 0.5em \usefont{OT1}{phv}{b}{sl} \color{DarkRed}} % Author font configuration

\author{Mingliang Liu, } % Your name

\postauthor{\footnotesize \usefont{OT1}{phv}{m}{sl} \color{Black} % Configuration for the institution name
Tsinghua University % Your institution

\par\end{flushleft}\HorRule} % Horizontal rule after the title

\date{} % Add a date here if you would like one to appear underneath the title block
%----------------------------------------------------------------------------------------

\maketitle % Print the title

\thispagestyle{fancy} % Enabling the custom headers/footers for the first page 

%----------------------------------------------------------------------------------------
%	ABSTRACT
%----------------------------------------------------------------------------------------

% The first character should be within \initial{}
\initial{D}\textbf{ynamic program slicing has been used in many applications.
Giri was a research project from UIUC, which implemented the dynamic backward slicing in LLVM.
It was selected as the Google Summer of Code 2013.
I did several improvements to Giri directed by Dr. Swarup:
\begin{inparaenum}[\itshape 1\upshape)]
\item Update the code to LLVM mainline and make it robust,
\item Reduce the trace size,
\item Make the Giri thread-aware (pthread only),
\item Improve the performance of run-time.
\end{inparaenum}
}

%------------------------------------------------
\section*{Introduction}
Dynamic program slice contains statements that influence the value of a variable occurrence for special program inputs~\cite{agrawal1990dynamic}.
The tradition program slicing was called static program slicing, which was firstly proposed by Weiser~\cite{weiser}.
There are many applications that use (or could benefit from) dynamic slicing, both by research and industry organizations (e.g. Microsoft, IBM).
For example, it's long been used in software debugging model~\cite{1993debugging,1999efficient} and testing~\cite{1993incremental}.
However, as far as I know, there is no public available dynamic slicing tool in GCC or Open64.

Sahoo et. al. from UIUC use dynamic program slicing to generate likely invariants for automated software fault localization~\cite{sahoo2013asplos}.
They implemented the dynamic program slicing code, called Giri, in LLVM compiler infrastructure for research purpose based on LLVM 2.6.
It also maps LLVM IR statements to source-level statements for its output using the debug metadata.  

%------------------------------------------------
\section*{Progress}
Our goal of GSoC is to make the Giri code update to the latest LLVM version and improve the performance, and/or reduce the tracing cost.
There are several things we did in this summer:
\begin{enumerate}
	\item \textbf{Update the code to LLVM mainline.}
		\begin{itemize}
			\item Release a v3.1 version which works with LLVM 3.1
			\item Make the code update to the latest LLVM code (3.4)
		\end{itemize}
	\item \textbf{Make the giri run-time library thread safe and the slicing thread aware}.
		The events of all processes and threads get thrown together in one trace file.
		Trace records indicate which thread is performing a particular operation using the thread id (\texttt{pthread\_t}).
		The trace files are written with lockes so that the race condition will not break the sequences.
	\item \textbf{Improving the giri run-time performance.}
		For example, fix the bug of \texttt{mmap} at the end of the tracing.
		We compute dynamically the cache size for how many trace records to hold in memory before flushing to disk.
	\item \textbf{Write dozens of unit tests and test several real programs.}
		There is also a simple test framework which will try all the unit tests at the top level of \texttt{giri/test/}.
	\item \textbf{Reduce the trace size.}
		We truncate the file at the end of tracer according to its real size.
	\item \textbf{Write documents for the project.}\\
		There are several sample pages:
		\begin{itemize}
			\item \href{https://github.com/liuml07/giri/wiki}{The Wiki Page Home}.
			\item \href{https://github.com/liuml07/giri/wiki/How-to-Compile-Giri}{How to Compile Giri}.
			\item \href{https://github.com/liuml07/giri/wiki/Hello-World}{Hello World!}
			\item \href{https://github.com/liuml07/giri/wiki/Example-Usage}{Example Usage With An Example}.
		\end{itemize}
\end{enumerate}

%------------------------------------------------
\section*{TODO List}
\begin{enumerate}
	\item Improve the performance of locking mechanism at the runtime
	\item Add tool/Tracer.cpp to the make list. It was removed when upgrading to LLVM 3.4
	\item Pass all the test programs of the test/ directory
	\item Try large test cases like nginx, squid, etc
	\item Consider more special function calls in TracingNoGiri::visitSpecialCall() function of the tracing pass
	\item Parallel the code writing the entry cache to trace file and adding entry to the cache.
\end{enumerate}

%------------------------------------------------
\section*{Contacts}
The Giri code is still under active development.
Dr. Swarup will direct me in the future to make the Giri code better.
We publish our code at \href{https://github.com/liuml07/giri}{https://github.com/liuml07/giri}~\cite{giri}.
For more information, please visit the homepage above.

%------------------------------------------------
\bibliographystyle{plain}
\bibliography{references} 

\end{document}
